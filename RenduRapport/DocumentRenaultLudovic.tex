\documentclass[a4paper, 12pt]{book}
\usepackage{graphicx}
\usepackage[french]{babel}
\usepackage[utf8]{inputenc}
\usepackage[T1]{fontenc}
\usepackage{multirow}
\usepackage{listings}
\usepackage{float}
\usepackage{url}
\usepackage[french]{algorithm}
\usepackage{algorithmic}
\usepackage{amsmath,amsfonts,amssymb}
\usepackage[nottoc, notlof, notlot]{tocbibind}


\begin{document}
\begin{titlepage}
  \begin{center}
    \begin{tabular*}{\textwidth}{l@{\extracolsep{\fill}}r}
      \includegraphics[height=1.5cm]{l1info.png}&
      \includegraphics[height=1.5cm]{licenceIV.png}
    \end{tabular*}
    \small 
    \rule{\textwidth}{.5pt}~\\
    \large 
    \textsc{Université Paris 8 - Vincennes à Saint-Denis}\vspace{0.5cm}\\
    \textbf{Licence Informatique - Outils Informatiques Collaboratifs}\vspace{3.0cm}\\
    \Large
    \textbf{Rendu Document}\vspace{1.5cm}\\
    \large
    \textbf{Ludovic \textsc{Renault}}\vspace{0.25cm}\\
    \large
    \textbf{Numero étudiant : 21001116}\vspace{1.5cm}\\

    Date de rendu : le 29/04/2022\vspace{1.75cm}\\
  \end{center}\vspace{1.5cm}~\\
  \begin{tabular}{ll}
    \hspace{-0.45cm}Tuteur -- Université~:~&~ Syrine \textsc{Saidi}\\
  \end{tabular}
\end{titlepage}

\frontmatter

\tableofcontents

\listoffigures

\listoftables

\mainmatter

\chapter{Figures}
\markboth{\sc Figures utilisées}{}

\subsection{Préambule}

Ce template est un template que j'utilise pour le rendu de mes rapports de projets. \footnote{ Il m'a été utile lors du premier semestre pour la matière méthologie de la programmation mais aussi lors du second pour la matière programmation impérative.}

\section{Logo de la licence IV}

\begin{figure}[htbp]
  \centering
  \includegraphics[width=0.2\linewidth]{licenceIV.png}
  \caption{logo de la licenceIV.\label{fig-bmp}}
\end{figure}

\section{Logo de la première année de licence}
\begin{figure}[htbp]
  \centering
  \includegraphics[width=0.3\linewidth]{l1info.png}
  \caption{logo de la première année de licence.\label{fig-bmp}}
\end{figure}

\subsection{réalisation du logo}
	Ce logo a été réalisé par un autre mais retouché par moi-même pour lui donner l'apparence de la L1.

\section{Les formules mathématiques}

 \begin{equation}
    \lim_{n \to \infty}
    \sum_{k=1}^n \frac{1}{k^2}
    = \frac{\pi^2}{6}
    \end{equation}

 
Nous avons aussi :

 $x^{2} \geq 0\qquad
    \text{pour tout } x\in\mathbf{R}$



\section{Les listes}

Nous pouvons donc grâce à ce template :
\begin{itemize}

    \item {Faire un rapport}
    \item {Faire un document plus simple}
\end{itemize}

\section{Citation}

Il y a une innocence dans l’admiration. C’est celle de l’homme qui n’envisage pas la possibilité que lui aussi pourrait être admiré un jour \cite{livre}


\section{Algorithme}

Comme algorithme nous pouvons avoir :

\begin{algorithm}
\caption{Calculate $y = x^n$}
\begin{algorithmic} 
\REQUIRE $n \geq 0 \vee x \neq 0$
\ENSURE $y = x^n$
\STATE $y \leftarrow 1$
\IF{$n < 0$}
\STATE $X \leftarrow 1 / x$
\STATE $N \leftarrow -n$
\ELSE
\STATE $X \leftarrow x$
\STATE $N \leftarrow n$
\ENDIF
\WHILE{$N \neq 0$}
\IF{$N$ is even}
\STATE $X \leftarrow X \times X$
\STATE $N \leftarrow N / 2$
\ELSE[$N$ is odd]
\STATE $y \leftarrow y \times X$
\STATE $N \leftarrow N - 1$
\ENDIF
\ENDWHILE
\end{algorithmic}
\end{algorithm}




\bibliographystyle{alpha}
\bibliography{Reference}



\end{document}